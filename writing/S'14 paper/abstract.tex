\begin{abstract}
  Supervised recursive neural network models (RNNs) for sentence
  meaning have been successful in an array of sophisticated language
  tasks, but it remains an open question whether they can learn
  compositional semantic grammars that support logical deduction.  We
  address this question directly by for the first time evaluating
  whether each of two classes of neural model --- plain RNNs and
  recursive neural tensor networks (RNTNs) --- can correctly learn
  relationships such as entailment and contradiction between pairs of
  sentences, where we have generated controlled data sets of sentences
  from a logical grammar.  Our first experiment evaluates whether
  these models can learn the basic algebra of logical relations
  involved. Our second and third experiments extend this evaluation to
  complex recursive structures and sentences involving quantification.
  We find that the plain RNN achieves only mixed results on all three
  experiments, whereas the stronger RNTN models generalizes well in
  every setting and appears capable of learning suitable
  representations for natural language logical inference.
\end{abstract}

% We address this question directly by for the first time evaluating whether each of two classes of neural model — plain RNNs and recursive neural tensor networks (RNTNs) — can correctly learn relationships such as entailment and contradiction between pairs of sentences, where we have generated controlled data sets of sentences from a logical grammar.
