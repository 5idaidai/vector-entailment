%
% File naaclhlt2015.tex
%

\documentclass[11pt,letterpaper]{article}
\usepackage{naaclhlt2015}
\usepackage{times}
\usepackage{latexsym}
\setlength\titlebox{6.5cm}    % Expanding the titlebox

%%% Custom additions %%%
\usepackage{hyperref}
\usepackage{url}
\usepackage[leqno, fleqn]{amsmath}
\usepackage{amssymb}
\usepackage{qtree}
\usepackage{graphicx}
\usepackage{booktabs}
\usepackage{colortbl}
\usepackage{caption}
\usepackage{subcaption}
\usepackage{xcolor}


\newcommand{\nateq}{\equiv}
\newcommand{\natind}{\mathbin{\#}}
%\newcommand{\natneg}{\raisebox{2px}{\tiny\thinspace$\wedge$\thinspace}}
\newcommand{\natneg}{\mathbin{^{\wedge}}}
\newcommand{\natfor}{\sqsubset}
\newcommand{\natrev}{\sqsupset}
\newcommand{\natalt}{\mathbin{|}}
\newcommand{\natcov}{\mathbin{\smallsmile}}

\newcommand{\plneg}{\mathop{\textit{not}}}
\newcommand{\pland}{\mathbin{\textit{and}}}
\newcommand{\plor}{\mathbin{\textit{or}}}

% Strikeout
\newlength{\howlong}\newcommand{\strikeout}[1]{\settowidth{\howlong}{#1}#1\unitlength0.5ex%
\begin{picture}(0,0)\put(0,1){\line(-1,0){\howlong\divide\unitlength}}\end{picture}}

\newcommand{\True}{\texttt{T}}
\newcommand{\False}{\texttt{F}}
\usepackage{stmaryrd}
\newcommand{\sem}[1]{\ensuremath{\llbracket#1\rrbracket}}

<<<<<<< HEAD
=======

\newcommand{\mynote}[1]{{\color{blue}#1}}

\newcommand{\tbchecked}[1]{{\color{red}#1}}

>>>>>>> FETCH_HEAD
\usepackage{gb4e}

\def\ii#1{\textit{#1}}
\newcommand{\word}[1]{\emph{#1}}

%%%%%%%%%%%%%%%%%%%%%%%%%%%%%%%%%%%%%%%%%%%%%%%%%%%%%%%%%%%%%%%%%%%%%%
%%%%% Code to simulate natbib's citealt, which prints citations with
%%%%% no parentheses:

\makeatletter
\def\citealt{\def\citename##1{{\frenchspacing##1} }\@internalcitec}
\def\@citexc[#1]#2{\if@filesw\immediate\write\@auxout{\string\citation{#2}}\fi
  \def\@citea{}\@citealt{\@for\@citeb:=#2\do
    {\@citea\def\@citea{;\penalty\@m\ }\@ifundefined
       {b@\@citeb}{{\bf ?}\@warning
       {Citation `\@citeb' on page \thepage \space undefined}}%
{\csname b@\@citeb\endcsname}}}{#1}}
\def\@internalcitec{\@ifnextchar [{\@tempswatrue\@citexc}{\@tempswafalse\@citexc[]}}
\def\@citealt#1#2{{#1\if@tempswa, #2\fi}}
\makeatother

%%%%%%%%%%%%%%%%%%%%%%%%%%%%%%%%%%%%%%%%%%%%%%%%%%%%%%%%%%%%%%%%%%%%%%


%%% %%%

\title{Recursive Neural Networks for Learning Logical Semantics}

%\Thanks{}}

%\author{Author 1\\
%	    XYZ Company\\
%	    111 Anywhere Street\\
%	    Mytown, NY 10000, USA\\
%	    {\tt author1@xyz.org}
%	  \And
%	Author 2\\
%  	ABC University\\
%  	900 Main Street\\
%  	Ourcity, PQ, Canada A1A 1T2\\
%  {\tt author2@abc.ca}}

\date{}

\begin{document}
\maketitle

\begin{abstract}
  Supervised recursive neural network models (RNNs) for sentence
  meaning have been successful in an array of sophisticated language
  tasks, but it remains an open question whether they can learn
  compositional semantic grammars that support logical deduction.  We
  address this question directly by for the first time evaluating
  whether each of two classes of neural model --- plain RNNs and
  recursive neural tensor networks (RNTNs) --- can correctly learn
  relationships such as entailment and contradiction between pairs of
  sentences, where we have generated controlled data sets of sentences
  from a logical grammar.  Our first experiment evaluates whether
  these models can learn the basic algebra of logical relations
  involved. Our second and third experiments extend this evaluation to
  complex recursive structures and sentences involving quantification.
  We find that the plain RNN achieves only mixed results on all three
  experiments, whereas the stronger RNTN models generalizes well in
  every setting and appears capable of learning suitable
  representations for natural language logical inference.
\end{abstract}

% We address this question directly by for the first time evaluating whether each of two classes of neural model — plain RNNs and recursive neural tensor networks (RNTNs) — can correctly learn relationships such as entailment and contradiction between pairs of sentences, where we have generated controlled data sets of sentences from a logical grammar.

\section{Introduction}\label{sec:intro}

Supervised recursive neural network models (RNNs) for sentence meaning
have been successful in a wide array of sophisticated language tasks,
including sentiment analysis \cite{socher2011semi,socher2013acl1},
image description \cite{sochergrounded}, and paraphrase detection
\cite{Socher-etal:2011:Paraphrase}. These results are encouraging
about the ability of these models to learn compositional semantic
grammars, but it remains an open question whether they can achieve the
same results as grammars based in logical forms
\cite{Warren:Pereira:1982,Zelle:Mooney:1996,ZetCol:2005,LiangJordan:2013} when it comes to core semantic
concepts like quantification, entailment, and contradiction. To date,
experimental investigations of these concepts using distributed
representations have been largely confined to short phrases
\cite{Mitchell:Lapata:2010,Grefenstette-etal:2011,baroni2012entailment,kalchbrenner2014convolutional}.
For robust natural language understanding, we must be able to model
these phenomena in their full generality on complex linguistic structures.

We address this question in the context of \ii{natural language
  inference} (also known as \ii{recognizing textual entailment};
\cite{dagan2006pascal}), in which the goal is to determine the core
inferential relationship between two sentences. Much of the
theoretical work on this task (and some successful implemented models
\cite{maccartney2009natural,watanabe2012latent}) involves \ii{natural
  logics}, which are formal systems that define rules of inference
between natural language words, phrases, and sentences without the
need of intermediate representations in an artificial logical
language. Following \cite{bowman2013can}, we use the natural logic
developed by \cite{maccartney2009extended} as our formal model. This
logic defines seven core relations of synonymy, entailment,
contradiction, and mutual consistency, as summarized in
Table~\ref{b-table}, and it provides rules of semantic combination for
projecting these relations from the lexicon up to complex phrases. The
formal properties and inferential strength of this system are now
well-understood \cite{Icard:Moss:2013,Icard:Moss:2013:LILT}.

In our experiments, we use this pre-specified logical grammar to
generate controlled data sets encoding the semantic relationships
between pairs of expressions and evaluate whether each of two
classes of neural model --- plain RNNs and recursive neural tensor
networks (RNTNs, \cite{socher2013acl1}) --- can learn those
relationships correctly. In our first experiment
(Section~\ref{sec:join}), we evaluate the ability of these models to
learn the core relational algebra of natural logic from data. Our
second experiment (Section~\ref{sec:recursion}) extends this
evaluation to cover relations between complex recursive structures
like $(a \plor b)$ and $\plneg(\plneg a \pland \plneg b)$, and our
third experiment (Section~\ref{sec:quantifiers}) involves relations
between quantified statements like \ii{every reptile walks} and \ii{every
turtle moves}. We find that the plain RNN achieves only mixed results
in all three experiments, whereas the stronger RNTN models generalized
well in every case, suggesting that it has in fact learned, or at
least learned to simulate, our target logical
concepts. % TODO: Update with current results

\begin{table}[htp]
  \centering
  \setlength{\tabcolsep}{15pt}
  \renewcommand{\arraystretch}{1.1}
  \begin{tabular}{l c l l} 
    \toprule
    Name & Symbol & Set-theoretic definition & Example \\ 
    \midrule
    entailment         & $x \natfor y$   & $x \subset y$ & \ii{turtle, reptile}  \\ 
    reverse entailment & $x \natrev y$   & $x \supset y$ & \ii{reptile, turtle}  \\ 
    equivalence        & $x \nateq y$    & $x = y$       & \ii{couch, sofa} \\ 
    alternation        & $x \natalt y$   & $x \cap y = \emptyset \wedge x \cup y \neq \mathcal{D}$ & \ii{turtle, warthog} \\ 
    negation           & $x \natneg y$   & $x \cap y = \emptyset \wedge x \cup y = \mathcal{D}$    & \ii{able, unable} \\
    cover              & $x \natcov y$   & $x \cap y \neq \emptyset \wedge x \cup y = \mathcal{D}$ & \ii{animal, non-turtle} \\ 
    independence       & $x \natind y$   & (else) & \ii{turtle, pet}\\
    \bottomrule
  \end{tabular}
  \caption{The entailment relations in $\mathfrak{B}$. 
    $\mathcal{D}$ is the universe of possible objects of the same type as those being compared, 
    and the relation $\natind$ applies whenever none of the other six do, including when there 
    is insufficient knowledge to choose a label.}
  \label{b-table}
\end{table}


% Citations to additional past work to be added.\\...\\...\\...\\...\\...

% TODO: Mention earlier work on NN interpretation - \cite{Garcez-etal:2001}


% Deep learning methods in NLP which learn vector representations for words have seen successful uses in recent years on increasingly sophisticated tasks \cite{collobert2011natural, socher2011semi, socher2013acl1, chen2013learning}. Given the still modest performance of semantically rich NLP systems in many domains---question answering and machine translation, for instance---it is worth exploring the degree to which learned vectors can serve as general purpose semantic representations. Much of the work to date analyzing vector representations for words (see \cite{baroni2013frege}) has focused on lexical semantic behaviors---like the similarity between words like \ii{Paris} and \ii{France}. Good similarity functions are valuable for many NLP tasks, but there are real use cases for which it is necessary to know how words relate to one another or to some extrinsic representation, and to model the ways in which word meanings combine to form phrase, sentence, or document meanings. This paper explores the ability of linguistic representations developed using supervised deep learning techniques to support interpretation and reasoning. 

% There are two broad family of tasks that could be used to test the ability of a model to develop general purpose meaning representations. In an interpretation task, sentences are mapped onto some denotation, such as  \ii{true} or \ii{false} for statements, or a factual answer for questions. There has been preliminary work in developing distributed models for interpretation \cite{grefenstette2013towards, rocktaschellow}, but developing a representation of world knowledge that corresponds accurately to the content expressed in language introduces considerable design challenges. I approach the problem by way of an inference task instead. Inferring the truth of one sentence from another does not require any preexisting knowledge representations, but nonetheless requires a precise representation of sentence meaning. I borrow the structure of the task from MacCartney and Manning  \cite{maccartney2009extended}. In it, the model is presented with a pair of sentences, and made to label the logical relation between the sentences as equivalence, entailment, or any of five other classes, as here:

%\begin{quote}
%\begin{enumerate}\small
%\item Many smartphone users avoid high bills overseas by disabling data service.
%\item Not everyone uses their smartphones for email when traveling abroad.
%\end{enumerate} 
%$\Rightarrow$ Entailment
%\end{quote}

%In this paper, we test the ability of recursive models to on three simple tasks, each of which is meant to capture a property that is necessary in representing natural language meaning in the setting of inference. I begin with an overview of MacCartney and Manning's \cite{maccartney2009extended} framework for inference, and of the recursive neural networks that I study. by showing that these models can learn to correctly represent entailment representations between sentences. I then show that these models can capture the meanings of recursive structures accurately up to a sufficient depth. I finally close with a brief demonstration of the ability of these models to reason over short natural language sentences involving quantifiers. 

% \subsection{Natural language inference and natural logic}

% In its simplest form, \ii{natural language inference} (also known as \ii{recognizing textual entailment}, as in \cite{dagan2006pascal}) is the task of determine whether one sentence entails another. Much of the theoretical work on this task (and some successful implemented models \cite{maccartney2009natural, watanabe2012latent}) involve \ii{natural logic}, formal systems that define sound rules of inference from one sentence of natural language to another without the need for intermediate representations in some other logic. The most powerful model that we are aware of for natural logic is due to MacCartney and Manning \cite{maccartney2009extended} and Icard \cite{icard2012inclusion}, and is centered around the definition of a set of seven logical relations which can hold between sentences, shown in Table \ref{b-table}.

% This approach to natural logic can capture much of the complexity of natural language meaning within a well understood framework, and is also fairly straightforward to implement in a machine learning setting since it can be reduced to a seven-way classification problem on sentence pairs. Our goal in this paper is to learn recursive neural network models which are able to mimic key behaviors of this system.

% \begin{table}
% \begin{center}
% \begin{tabular}{|c|c|c|c|} \hline
% name & symbol & set-theoretic definition & example \\ \hline \hline
% entailment & $x \sqsubset y$ & $x \subset y$ & \ii{crow, bird}  \\ \hline
% reverse entailment & $x \sqsupset y$ & $x \supset y$ & \ii{Asian, Thai}  \\ \hline
% equivalence & $x \equiv y$ & $x = y$ & \ii{couch, sofa} \\ \hline
% alteration & $x$ $|$ $y$ & $x \cap y = \emptyset \wedge x \cup y \neq \mathcal{D}$ & \ii{cat, dog} \\ \hline
% negation & $x \natneg y$ & $x \cap y = \emptyset \wedge x \cup y = \mathcal{D}$ & \ii{able, unable} \\ \hline
% cover & $x \smallsmile y$ & $x \cap y \neq \emptyset \wedge x \cup y = \mathcal{D}$ & \ii{animal, non-ape} \\ \hline
% independence & $x$ \# $y$ & (else) & \ii{hungry, hippo}\\ \hline
% \end{tabular}
% \caption{The entailment relations in  $\mathfrak{B}$. $\mathcal{D}$ is the universe of possible objects of the same type as those being compared, and the relation \# applies whenever none of the other six do, including when there is insufficient knowledge to choose a label.}
% \label{b-table}
% \end{center}
% \end{table}

% The goal of each of the three experiments that we propose is to learn classifiers that are able to classify pairs of statements from some highly constrained language into these seven classes. It should be noted that this is not a balanced classification problem. If arbitrary pairs of statements are chosen for comparison in almost any domain of natural language, the minimally informative \# relation will apply between them. % TODO: How much should we say about this?


% The natural logic engine at the core of MacCartney and Manning's system requires a complex set of linguistic knowledge, much of which takes the form of what he calls projectivity signatures. These signatures are tables showing the entailment relation that must hold between two strings that differ in a given way (such as the substitution of the argument of some quantifier), and are explicitly provided to the model
%for dozens of different cases of insertion, deletion and substitution of different types of lexical item. For example in judging the inference \ii{no animals bark $|$ some dogs bark} it would first used stored knowledge to compute the relations introduced by each of the two differences between the sentences. Here, the substitution of \ii{no} for \ii{some}  yields $\natneg$ and the substitution of \ii{dogs} for \ii{animals} yields $\sqsupset$. It would then use an additional store of knowledge about relations to resolve the resulting series of relations into one ($|$) that expresses the relation between the two sentences being compared:
%\begin{quote}

%1. \ii{no animals bark $\natneg$ \textbf{some} animals bark}\\
%2. \ii{some animals bark $\sqsupset$ some \textbf{dogs} bark}\\
%3. \ii{no animals bark $[\natneg\bowtie\thinspace\sqsupset\thinspace = |]$ some dogs bark}

%\end{quote}

% Work to date on inference in neural network models is quite limited.
% \citet{baroni2012entailment} have achieved limited success in building a classifier to judge entailments between one- and two-word phrases (including some with quantifiers), though their vector representations were crucially based on distributional statistics and were not  learned for the task.
% In another line of research, \citet{garrette2013formal} propose a way to improve standard discrete NLI with vector representations. They propose a deterministic inference engine (similar to MacCartney's) which is augmented by a probabilistic component that evaluates individual lexical substitution steps in the derivation using vector representations, though again these representations are not learned, and no evaluations of this system have been published to date.
% \label{sec2}


\subsection*{Neural network models for relation classification} \label{methods}

% TODO: Should we cite our NIPS manuscript?

We follow the approach to learning semantically meaningful embeddings 
proposed in \citet{bowman2013can}, which is centered on the problem of
labeling a pair of words or sentences with one of a small set of logical
relations. The architecture of the model that we use, which is limited
to only pairs of single symbols (such as words), is depicted in
Figure~\ref{sample-figure}. The model represents the two input symbols
as embeddings, which are fed into a comparison function based on one
of two types of neural network layer functions to produce a representation
for the relationship between the two symbols. This representation is then
fed into a simple softmax classifier which outputs a distribution over
possible labels. The entire network, including the embeddings, are trained
through backpropagation with AdaGrad \cite{duchi2011adaptive}.

\begin{figure}[tp]
  \centering
  \footnotesize

\newcommand{\labeledtreenode}[4][3.5]{\put(#2){\makebox(0,0){{\fcolorbox{black}{#4}{\makebox(#1,0.3){#3}}}}}}

\newcommand{\textlabel}[4][3.5]{\put(#2){\makebox(0,0){{\fcolorbox{white}{white}{\makebox(#1,0.3){#3}}}}}}

\definecolor{lexcolor}{HTML}{F5F7C4}
\definecolor{compositioncolor}{HTML}{BBEBFF}
\definecolor{comparisoncolor}{HTML}{FFC895}
\definecolor{softmaxcolor}{HTML}{A5FF8A}


\setlength{\unitlength}{0.61cm}
\begin{picture}(21,7.5)
  
  \labeledtreenode[2.4]{11.5,7}{$P(\sqsubset) = 0.8$}{softmaxcolor}  
  \put(11.5,5.7){\vector(0,1){1}}  
  \labeledtreenode[7.85]{11.5,5.4}{all reptiles walk \emph{vs.}~some turtles move}{comparisoncolor}


  \textlabel{8,7}{Softmax classifier}{black}
  \textlabel{4.5,5.4}{Comparison N(T)N layer}{black}
      
  \textlabel{11.75,3.6}{Composition RN(T)N layers}{black}

  \textlabel{5,0.1}{Learned, randomly initialized word vectors}{black}
  
  %%%%%%%%%%%%%%%%%%%%%%%%%%%%%%%%%%%%%%%%%%%%%%%%%%
    
  \put(1.75,1.35){\vector(2,1){1.7}}
  \labeledtreenode{1.75,1}{all}{lexcolor}

  \put(6,1.35){\vector(-2,1){1.7}}
  \labeledtreenode{6,1}{reptiles}{lexcolor}

  \put(4,2.75){\vector(2,1){1.7}}
  \labeledtreenode{4,2.5}{all reptiles}{compositioncolor}

  \put(8.25,2.75){\vector(-2,1){1.7}}
  \labeledtreenode{8.25,2.5}{walk}{lexcolor}

  \put(6.25,4.25){\vector(4,1){3.25}}
  \labeledtreenode{6.25,3.9}{all reptiles walk}{compositioncolor}
  
  %%%%%%%%%%%%%%%%%%%%%%%%%%%%%%%%%%%%%%%%%%%%%%%%%%%

  \put(12.75,1.35){\vector(2,1){1.7}}
  \labeledtreenode{12.75,1}{some}{lexcolor}

  \put(17,1.35){\vector(-2,1){1.7}}
  \labeledtreenode{17,1}{turtles}{lexcolor}

  \put(15,2.75){\vector(2,1){1.7}}
  \labeledtreenode{15,2.5}{some turtles}{compositioncolor}

  \put(19.25,2.75){\vector(-2,1){1.7}}
  \labeledtreenode{19.25,2.5}{move}{lexcolor}
          
  \put(17.25,4.25){\vector(-4,1){3.25}}
  \labeledtreenode{17.25,3.9}{some turtles move}{compositioncolor}
  
\end{picture}



  \caption{The model structure used to compare \ii{turtle} and \ii{animal}. 
    The same structure is used for both the RNN and RNTN layer functions.} 
  \label{sample-figure}
\end{figure}

For a comparison funtion, we evaluate versions of the model with both a plain neural
network (NN) layer function and a neural tensor network (NTN) layer function
\eqref{rntn} proposed in \citet{chen2013learning}. A leaky ReLU
nonlinearity \cite{maasrectifier} is applied to the output of either
layer function.
%
\begin{gather} \label{rnn}
\vec{y}_{\textit{NN}} = f(\mathbf{M} [\vec{x}^{(l)}; \vec{x}^{(r)}] + \vec{b}) \\ % TODO: Add column vectors?
\label{rntn}
\vec{y}_{\textit{NTN}} = f(\vec{x}^{(l)T} \mathbf{T}^{[1 \ldots n]} \vec{x}^{(r)} + \mathbf{M} [\vec{x}^{(l)}; \vec{x}^{(r)}] + \vec{b})
\end{gather} 
%
Here, $\vec{x}^{(l)}$ and $\vec{x}^{(r)}$ are the column vector
representations for the left and right children of the node, and
$\vec{y}$ is the node's output.  The RNN concatenates them, multiplies
them by an $n \times 2n$ matrix of learned weights, and applies the
element-wise non-linearity to the resulting vector. The RNTN has the
same basic structure, but with the addition of a learned third-order
tensor $\mathbf{T}$, dimension $n \times n \times n$, modeling
multiplicative interactions between the child vectors. Both models
include a bias vector~$\vec{b}$.


This model differs from that of \citet{bowman2013can} in two ways. Because 
the inputs are single symbols, there is no need for the composition functions
which are used in that (and prior) work. Also, for our second experiment on 
WordNet data, we introduce a new neural network layer between the embedding input
and the comparison function, which is meant to facilitate initializing the embeddings
from an outside source, and was found to help peformance in that setting.

%\ii{Source code and generated data will be released after the conclusion of the review period.} % TODO: Or upon request? Attach anonymized code?


\section{Relation composition}\label{sec:join}

If a model is to learn the behavior of a relational logic like the one
presented here from a finite amount data, it must be able to learn to
deduce new relations from seen relations in a sound
manner. The simplest such deductions involve atomic statements using
the relations in Table~\ref{b-table}. For instance, given that $a
\natrev b$ and $b \natrev c$, one can conclude that $a \natrev c$, by
basic set-theoretic reasoning (transitivity of $\natrev$). Similarly,
from $a \natfor b$ and $b \natneg c$, it follows that $a \natalt c$.
The full set of sound inferences of this form is summarized in
Table~\ref{tab:jointable}; cells containing a dot correspond to pairs
of relations for which no valid inference can be drawn in our logic.

% about the relations themselves that do not depend on the
% internal structure of the things being compared. For example, given
% that $a\sqsupset b$ and $b\sqsupset c$ one can conclude that
% $a\sqsupset c$ by the transitivity of $\sqsupset$, even without
% understanding $a$, $b$, or $c$. These seven relations support more
% than just transitivity: MacCartney and Manning's
% \cite{maccartney2009extended} join table defines 32 valid inferences
% that can be made on the basis of pairs of relations of the form $a R
% b$ and $b R' c$, including several less intuitive ones such as that if
% $a \natneg b$ and $b~|~c$ then $a \sqsupset c$.

\begin{table}[htp]
  \centering  
  \setlength{\arraycolsep}{8pt}
  \renewcommand{\arraystretch}{1.1}
  \newcommand{\UNK}{\cdot}  
  $\begin{array}[t]{c@{ \ }|*{7}{c}|}
    %\hline
    \multicolumn{1}{c}{}
             & \nateq     & \natfor     & \natrev     & \natneg    & \natalt     & \natcov     & \multicolumn{1}{c}{\natind} \\
    \cline{2-8}
    \nateq  & \nateq &   \natfor &  \natrev &  \natneg &   \natalt &  \natcov &  \natind \\
    \natfor & \natfor &  \natfor &  \UNK &  \natalt &   \natalt &  \UNK &  \UNK \\
    \natrev & \natrev &  \UNK &  \natrev &  \natcov &   \UNK &  \natcov &  \UNK \\
    \natneg & \natneg &  \natcov &  \natalt &  \nateq &    \natrev &  \natfor &  \natind \\
    \natalt & \natalt &  \UNK &  \natalt &  \natfor &   \UNK &  \natfor &  \UNK \\
    \natcov & \natcov &  \natcov &  \UNK &  \natrev &   \natrev &  \UNK &  \UNK \\
    \natind & \natind & \UNK &  \UNK &  \natind &  \UNK &  \UNK &  \UNK \\
    \cline{2-8}
  \end{array}$
  \caption{Full set of inferences between natural logic relations.
    Given $a \mathbin{R} b$ and $b \mathbin{S} c$, where $R$ and $S$
    are the row and column relations, respectively, and $a$, $b$, and
    $c$ are arbitrary formulae, the table provides the relation
    $a \mathbin{T} c$.}
  \label{tab:jointable}
\end{table}

To test our systems's ability to learn this relational structure, we
create small models for our logic in which terms denote sets of
entities from a single domain of seven entities (integers).
Figure~\ref{lattice-figure} depicts a small model of this form. The
lattice structure gives the full model, for which all the statements on
the right are valid. We divide these statements evenly into training and
test sets, and remove from the test set those statements which cannot be 
proven from the training statements using the logic depicted in 
Figure~\ref{lattice-figure}.

\begin{figure}[htp]
  \centering
  \begin{subfigure}[t]{0.4\textwidth}
    \centering
    \newcommand{\labelednode}[4]{\put(#1,#2){\oval(1.5,1)}\put(#1,#2){\makebox(0,0){$\begin{array}{c}#3\\\{#4\}\end{array}$}}}
    \setlength{\unitlength}{1cm}
    \begin{picture}(5,5.5)
      \labelednode{2.50}{5}{}{1,2,3}
      
      \put(0.75,4){\line(3,1){1.5}}
      \put(2.5,4){\line(0,1){0.5}}
      \put(4.25,4){\line(-3,1){1.5}}
      
      \labelednode{0.75}{3.5}{a,b}{1,2}
      \labelednode{2.50}{3.5}{c}{1,3}
      \labelednode{4.25}{3.5}{d}{2,3}
      
      \put(0.75,2.5){\line(0,1){0.5}}
      \put(0.75,2.5){\line(3,1){1.5}}
      
      \put(2.5,2.5){\line(-3,1){1.5}}
      \put(2.5,2.5){\line(3,1){1.5}}
      
      \put(4.25,2.5){\line(0,1){0.5}}
      \put(4.25,2.5){\line(-3,1){1.5}}
      

      \labelednode{0.75}{2}{e,f}{1}
      \labelednode{2.50}{2}{}{2}
      \labelednode{4.25}{2}{g,h}{3}
      
      \put(2.5,1){\line(-3,1){1.5}}
      \put(2.5,1){\line(0,1){0.5}}
      \put(2.5,1){\line(3,1){1.5}}
      
      \labelednode{2.5}{0.5}{}{}
    \end{picture}
    \caption{Simple boolean model. The letters name the sets. Not all sets have names, and
    some sets have multiple names, so that learning $\nateq$ is non-trivial.}
  \end{subfigure}
  \qquad
  \begin{subfigure}[t]{0.5\textwidth}
    \centering
    \setlength{\tabcolsep}{12pt}
    \begin{tabular}[b]{c  c}
      \toprule
      Train & Test \\
      \midrule
                    & $b \nateq b$ \\
      $b \natcov c$ &               \\
                    & $b \natcov d$ \\
                    & \strikeout{$b \natrev e$} \\
      $c \natcov d$ &               \\
      $c \natrev e$ &               \\
                    & \strikeout{$c \nateq f$} \\
      $c \natrev g$ &               \\ 
                    & $e \natfor b$ \\
      $e \natfor c$ &               \\[-1ex]
      $\vdots$      & $\vdots$ \\
      \bottomrule
    \end{tabular}

    \caption{An example train/test split from the full set
      of statements one can make about the model.
      Statements not provable from the test data are crossed out.}
  \end{subfigure}  
  \caption{Some sample randomly generated sets, and some of the relations defined between them.}
  \label{lattice-figure}
\end{figure} 
% TODO: Recheck strikethrough distribution.

% We test the model's ability to learn this behavior by creating
% artificial data sets of terms which represent sets of numbers. Since
% MacCartney and Manning's set of relations hold between sets as well as
% between sentences, we can use the underlying set structure to generate
% the all of the relations that hold between any pair of these terms, as
% in Figure \ref{lattice-figure}. We train the model defined above on a
% subset of these relations, but rather then presenting the model with a
% pair of tree-structured sentences as inputs, simply present it with
% two single terms, each of which corresponds to a single vector in the
% (randomly initialized) vocabulary matrix $V$, ensuring that the model
% has no information about the terms being compared except the relations
% between them.

In our experiments, we create 80 randomly generated sets drawing from
a domain of seven elements. This yields a data set consisting of
6400 statements about pairs of entities. 3200 of these pairs are
chosen as a test set, and that test set is further reduced to the 2960
examples that can be provably derived from the training data.

We trained versions of both the RNN model and the RNTN model on these
data sets. In both cases, the models were implemented exactly as
described in Section~\ref{methods}, but since the items being compared
are single terms rather than full tree structures, the composition
layer was not involved, and the two models differed only in which
layer function was used for the comparison layer. We simply present
the models with two single terms, each of which corresponds to a
single vector in the (randomly initialized) vocabulary matrix $V$,
ensuring that the model has no information about the terms being
compared except the relations between them. 

We found that the RNTN model worked best with 11-dimensional vector representations for the
80 sets and a 90-dimensional feature vector for the classifier. This
model was able to correctly label 99.3\% of the provably derivable test
examples, and 99.1\% of the remaining test examples. The simpler RNN model
worked best with 11 and 75 dimensions, respectively, but was able to
achieve accuracies of only 90.0\% and 87.0\%, respectively.

% RNTN log: tue-j-11-6x80-hip.txt
% Train PER: 0.0021875

% RNN log: tue-j-11-6x80-r-75.txt
% Train PER: 0.047188

% and create a dataset consisting of the relations between every pair of
% sets, yielding 6400 pairs. 3200 of these pairs are then chosen as a
% test dataset, and that test dataset was further split into the 2960
% examples that can be provably derived from the test data using
% MacCartney and Manning's join table (or by the symmetry of the
% relations in about half of the cases) and the 240 that
% cannot. % TODO: Say more about symmetry?

% We tested a version of both the RNN model and the RNTN model on these
% data. In both cases, the models were implemented exactly as described
% in Section~\ref{methods}, but since the items being compared are
% single terms rather than full tree structures, the composition layer
% was not used, and the two models differed only in which layer function
% was used for the comparison layer. We found that the RNTN model worked
% best with 11 dimensional vector representations for the 80 sets and a
% 90 dimensional feature vector for the classifier. This model was able
% to correctly label 99.3\% of the derivable test examples, and 99.1\%
% of the remaining examples. The simpler RNN model worked best with 11
% and 75 dimensions, respectively, but was able to achieve accuracies of
% only 90.0\% and 87.\%, respectively.

% TODO: T-sne the RNTN model's embeddings.

These results are fairly straightforward to interpret. The RNTN model
was able to accurately encode the relations between the terms in the
geometric relations between their vectors, and was able to then use
that information to recover relations that were not overtly included
in the training data. In contrast, the RNN model was able to
approximate this behavior only incompletely. It is possible but not
likely that it could be made to find a good solution with further
optimization on different learning algorithms, or that it would do
better on a larger universe of sets for which there would be a larger
set of training data to learn from, but the RNTN is readily able to achieve
these effects in the setting discussed here.


\section{Recursive structure}

\begin{figure}[t]
\begin{center}
\begin{tabular}{lll}
$a\equiv a$		&~~~&	$(c~(and~(not~d)))~\#~f$\\
$b~\#~c$			&~~~&	$(not~(c~(or~b)))~\sqsubset~(not~c)$\\
$d\natneg(not~d)$	&~~~&	$f~\#~((c~(or~(not~d)))~(and~a))$\\
$(c~(and~d))\sqsubset d$&~~~&$d\sqsupset((d~(or~d))~(and~(not~b)))$\\
\end{tabular}
\end{center}

\caption{Some sample randomly generated pairs of propositional logic statements.  \label{prop-figure}} 
\end{figure}

% TODO: Cite Chomsky/Hauser/Fitch?

Recursive structure is a prominent  feature of natural language. Consider, for example, \ii{Alice said hello}, \ii{Bob said that Alice said hello}, and \ii{Carl thinks that Bob said that Alice said hello}. Overt recursion of this kind is easy to find, and theoretical accounts of natural language syntax and semantics rely heavily on recursive structures.
In order for a model to be able to accurately learn natural language meanings, then, we expect that it would need to be able to learn to represent the meanings of function words in a such a way that they are able to behave correctly when taking their own outputs as input.

% Goal: Deep recursion!

We again test this phenomenon within the framework of MacCartney and Manning-style entailment reasoning, but we replace the unanalyzed symbols from the previous experiment with expressions that involve recursive structure. To define these expressions, we turn to propositional logic, a relatively simple logic in which each variable represents either \ii{true} or \ii{false}. We generate data of the form seen in Figure \ref{prop-figure}: strings of arbitrary length consisting of six elementary proposition variables and the operators \ii{and}, \ii{or}, and \ii{not}, arranged in pairs with the logical relations between them specified. 



% TODO: Worth explicitly calling this project theorem proving? Yes! There was some confusion at CSLI.

Socher et al. \cite{socher2012semantic} have previously demonstrated the learning of a logic in a matrix-vector RNN model somewhat similar to our own, but the logic discussed here is substantially stronger, and a much better approximation of the kind of structure that is needed for natural language. The logic learned in that experiment is boolean, wherein the atomic symbols are simply the values 0 and 1, rather than variables over those values. While learning the operators of that logic is not trivial, the ouptuts of each operator can be represented accurately by a single bit. The statements of propositional logic learned here describe conditions on the truth values of propositions where those truth values are not known. As opposed to the two-way contrasts seen in \cite{socher2012semantic}, this logic distinguishes between 64 (2^6) possible assignments of truth values, and expressions of this logic define arbitrary conditions on these possible assignments, for a total of 2^{64} ($\approx 10^{20}$) possible statements that the recursive model needs to be able to distinguish. To frame this distinction in another way, the relational statements in our data our theorems about the relations between statements in the logic tested in \cite{socher2012semantic}.

We randomly generate pairs of parentheses-bracketed statements of the logic and then randomly divide the results into training and test data sets. To compute the relation between each pair of statements, we exhaustively enumerate the sets of assignments of truth values to proposition variables that would satisfy each of the statements and then convert the set-theoretic relation between those assignments into one of the seven relations. 
If we do not implement any constraint that the two statements being compared are similar in any way, the generated data consists in large part of statements in which the two refer to largely separate subsets of the six variables, and to which we will nearly always assign the \# relation. In an effort to balance the distribution of relation labels without departing from the basic task of modeling propositional logic, we disallow individual pairs of statements from referring to more than four of the six proposition variables. 

We deduplicate the data and discard pairs in which either statement is a tautology or contradiction (a statement that is true of either all or no possible assignments), for which none of the seven relation labels can accurately apply. We then divide the generated pairs into size bins based on the number of logical operators (\ii{and}, \ii{or}, or \ii{not}) in the larger of the two pairs being compared, and discard examples of size greater than twelve by this measure. Finally, we randomly sample 15\% of each bin for a held out test set.
...
65k train  

 compute the relaitons between them, discard  , and consist of about 248k training examples and 44k test examples.

 Each of the six symbols and each of the three operators is treated as a word for the purposes of our model, and is represented by a randomly initialized vector representation.





...


% TODO: Add figure for results

train up to depth 5
test up to depth 12

% Describe data

\begin{figure}[t]
\begin{center}
...\\
...\\
...\\
...\\
...\\
...\\
...\\
...\\
...\\
\end{center}

\caption{Model performance by expression size. The vertical dashed line indicates the size of the largest expressions included in the training data.  \label{prop-figure}} 
\end{figure}


The RNN model was approximately optimal with N dimensional word representations and an M dimensional comparison layer. The RNTN was approximately optimal with N and M dimensions, respectively.

% TODO: Include dimensionality


\section{Reasoning with natural language quantifiers and negation}\label{sec:quantifiers}

We have seen that the RNTN can learn an approximation of propositional
logic.  However, natural languages can express functional meanings of
considerably greater complexity than this.  As a first step towards
investigating whether our models can capture this complexity, we now
attempt to directly measure the degree to which RNNs are able to
develop suitable representations for the semantics of natural language
quantifiers like \ii{some} and \ii{all}. Quantification is far from
the only place in natural language where complex functional meanings
are found, but it is a natural starting point, since it can be tested
in sentences whose structures are otherwise quite simple, and since it
has formed a standard case study in prior formal work on natural
language inference.

% \subsection{Data}

This experiment replicates similar work described in
\cite{bowman2013can}, which found that RNTNs can learn to reason well
with quantifier meanings given sufficient training data. This paper
replaces the partially manually annotated data in that paper with data
that is generated directly using the logical system that we hope to
model, yielding results that we believe to be substantially more
straightforward to interpret.

\paragraph{Experiments}
Our experimental data consist of pairs of sentences generated from a
small artificial grammar. Each sentence contains a quantifier, a noun
which may be negated, and an intransitive verb which may be
negated. We use the basic quantifiers \ii{some}, \ii{most}, \ii{all},
\ii{two}, and \ii{three}, and their negations \ii{no}, \ii{not-all},
\ii{not-most}, \ii{less-than-two}, and \ii{less-than-three}. We also
include five nouns, four intransitive verbs, and the negation symbol
\ii{not}. In order to be able to define relations between sentences
with differing lexical items, we define the lexical relations between
each noun--noun pair, each verb--verb pair, and each
quantifier--quantifier pair. The grammar accepts aligned pairs of
sentences of this form and calculates the natural logic relationship
between them.  Some examples of these data are provided in
Table~\ref{examplesofdata}.  As in previous sections, the goal of
learning is then to assign these relational labels accurately to
unseen pairs of sentences.

%nouns = ['warthogs', 'turtles', 'mammals', 'reptiles', 'pets']
%verbs = ['walk', 'move', 'swim', 'growl']
%dets = ['all', 'not_all', 'some', 'no', 'most', 'not_most', 'two', 'lt_two', 'three', 'lt_three']
%adverbs = ['', 'not']

% To assign relation labels to sentence pairs, we built a small
% task-specific implemenation of MacCartney's logic that can
% accurately label sentences of this restricted language. The logic is
% not able to derive all intuitively true relations of this language,
% and fails to derive a single unique relation for certain types of
% statement, including De Morgean's laws (e.g. \ii{(all pets) growl
% $\natneg$ (some pet) (not growl)}), and we simply discard these
% examples. Exhaustively generating the valid sentences under this
% grammar and choosing those to which a relation label can be assigned
% yields 66k sentence pairs. Some examples of these data are provided
% in Table~\ref{examplesofdata}.

\begin{table}[htp]
  \centering
  \begin{tabular}{l c l}
    \toprule
    (most warthogs) walk         & $\natneg$ & (not-most warthogs) walk\\
    (most mammals) move          & $\natind$ &  (not-most (not turtles)) move\\
    (most (not pets)) (not swim) & $\natrev$ & (not-most (not pets)) move 
    \\[2ex]    
    (no turtles) (not growl)     & $\natalt$ & (no turtles) (not swim)\\
    (no warthogs) swim           & $\natrev$ & (no warthogs) move\\
    (no warthogs) move           & $\natfor$ & (no (not reptiles)) swim\\
    \bottomrule
  \end{tabular}
  \caption{Sample data involving two different quantifier pairs.}
  \label{examplesofdata}
\end{table}

We evaluate the model using two experimental settings. In the simpler
setting, \textsc{all split}, we randomly sample 85\% of the data and evaluate on the
remaining 15\%. In this setting, the model is being asked to learn a
complete reasoning system for the limited language and logic presented
in the training data, but it is not being asked to generalize to test
examples that are substantially different from those it was trained
on. Crucially though, to succeed on this task, the model must be able
to recognize all of the lexical relations between the nouns, verbs,
and quantifiers and how they interact. For instance, it might see
\eqref{p1} and \eqref{p2} in training and, from that information,
determine \eqref{p3}.
%
% do not allow a blank line --- adds too much space
%
\begin{gather}
  \text{(most turtle) swim} \natalt \text{(no turtle) move}\label{p1}
  \\
  \text{(all lizard) reptile} \natfor  \text{(some lizard) animal}\label{p2}
  \\
  \text{(most turtle) reptile} \natalt \text{(no turtle) animal}\label{p3}
\end{gather}
%
% do not allow a blank line --- adds too much space
%
While our primary interest is in discovering the extent to which our
models can learn to encode the logic given an arbitrary amount of
data, we are also interested in the degree to which they can infer a
correct representation for the logic from more constrained training
data. To this end, we segment the sentence pairs according to which
quantifiers appear in each pair, and then hold out one such pair for
testing. We hypothesize that a model that can efficiently learn to
represent a logic should be able to construct an accurate
representation of each held out quantifier from the way that it
interacts with the other nine quantifiers which are not held
out. Since running this experiment requires choosing a pair of
quantifiers to hold out before training, the resource demands of
training prevent us from testing each of the 55 possible possible
pairs of quantifiers, and we choose only four pairs to test on.  Three
of these (\ii{two}/\ii{less-than-two}, \ii{not-all}/\ii{not-most}, and
\ii{all}/\ii{some}) were chosen because they allow for the most
different class labels at in the training data. The fourth is a
self-pair (\ii{no}/\ii{no}), meant to test that the model correctly
handles equality.

\begin{table}[tp]
  \centering
  \begin{tabular}{ l rrr }
    \toprule
    Data & Most frequent class & 16 dim RNN  & 20 dim RNTN\\
    \midrule
    \textsc{all split}	& 35.4 (7.5) &	67.4 (56.5)&	\textbf{100.0 (100.0)}
    \\[1ex]    
    \textsc{pair two/less-than-two}	& 59.8 (18.4) &	77.2 (57.4) &	\textbf{100.0 (100.0)} \\
    \textsc{pair not-all/not-most}	&0 (0) & 66.0 (54.3) &	\textbf{93.8 (91.5)} \\
    \textsc{pair all/some}	& 0 (0) & 62.7 (65.9)  &	\textbf{78.4 (84.7)} \\
    \textsc{pair no/no}	& 30.8 (10.2) &	67.8 (61.8) &	\textbf{99.9 (99.7)} \\
    \bottomrule
  \end{tabular}
  \caption{Performance on the quantifier experiments. Results are reported as accuracy scores followed by macroaveraged F1 scores in parentheses.}
  \label{resultstable}
\end{table} 


\paragraph{Results} 
The results for these experiments are shown in
Figure~\ref{resultstable}. We compare the results to a most frequent
class baseline, which reflects the frequency in the test data of most
frequent class in the training data, $\natind$.  After some
cross-validation, we chose 16 and 20-dimensional
representations for the RNN and RNTN respectively, and 75 dimensional 
feature vectors for the classifier.

The RNN performed poorly at this task, even though the sentences used
in these examples are short enough to avoid the pathology shown in
Figure~\ref{prop-falloff}.  However, the perfect performance by the
RNTN on the \textsc{all split} and \textsc{pair two/less-than-two} 
experiment and its strong performance on \textsc{pair no/no} suggests
that that this stronger model is able to handle quantifiers correctly
given sufficient training data, but the weaker results on two of the
training settings suggest that it may not be able to generalize from
data this impoverished in general. However, though the question of how
much data is necessary to accurately capture quantifier behavior in a
na\"ive model remains open, the fact that both models perform far
above baseline is promising.
\section{The SICK textual entailment challenge}

The tree pair model architecture that we use is novel, and though the underlying recursive approach has been validated elsewhere, there is no guarantee that this architecture is suitable for handling inference in the face of the noisy labels and diverse range of linguistic structures seen in typical natural language data, and our experiments on artificial data are not alone sufficient to address this issue. To investigate the ability of the model to learn on natural data, we train a version of our model on the SICK textual entailment challenge corpus \cite{marelli2014sick}. The corpus consists of about 10k natural language sentence pairs (generated using data expansion from a smaller set), labeled with \ii{entailment}, \ii{contradiction}, or \ii{neutral}. At only a few thousand distinct sentences (many of them variants on an even smaller set of template sentences), the corpus is not large enough to train a high quality learned model of general natural language, but it is the largest hand labeled entailment corpus that we are aware of.

\begin{table*}[htp]
  \centering\small
  \begin{tabular}{lcl}
    \toprule
A woman is not boiling shrimps.& \ii{contradiction}&	A woman is boiling shrimps.\\
A wild deer is jumping a fence. &\ii{entailment}	&A deer is jumping a fence.\\
A potato is being sliced by a woman. &\ii{entailment}	&A woman is slicing a potato.\\
The plane, which is south African, is flying in a blue sky.& \ii{entailment}&	An airplane is flying through the air.\\
    \bottomrule
  \end{tabular}
  \caption{\label{examplesofsickdata}Examples of successful classifications on SICK.}
\end{table*}

Adapting to this task required us to make a few additions to the techniques discussed in Section \ref{methods}. In order to better handle rare words, we initialized our word embeddings using the 200 dimensional vectors trained with 
GloVe \cite{pennington2014glove} on data from Wikipedia. Since 200 dimensional vectors are too large to be practical in an RNTN on a small dataset, a new embedding transformation layer is needed. Before any embedding is used as an input to a recursive layer, it is passed through an additional $\tanh$ neural network layer with the same output dimension as the recursive layer which aggregates any usable information from the embedding vectors into a more compact working representation.

We also supplemented the SICK training data\footnote{We tuned the model using performance on a held out development set, but report performance here for a version of the model trained on both the training and development data and tested on the SICK test set. We also report training accuracy on a small sample from each data source.} with 600k examples of entailment data from the Denotation Graph project (DG, \citealt{hodoshimage}), a corpus of automatically labeled (and thus noisy) entailment examples over image captions, the same genre of text from which SICK was drawn. We trained a single model on data from both sources, but used a separate set of softmax parameters for classifying into the labels from each source. We parsed the data from both sources with the Stanford PCFG Parser v. 3.3.1 \cite{klein2003accurate}. We also found that we were able to train a working model much more quickly with an additional technique: we collapse subtrees that were identical across both sentences in a pair down to a single head word. The test data on which we report performance are collapsed in this way, and both collapsed and uncollapsed copies of the training data are used in training. Finally, in order to improve regularization on the noisier data, we used dropout \cite{srivastava2014dropout} at the input to the comparison layer (10\%) and at the output from the embedding transform layer (25\%). 

\begin{table}[tp]
  \centering \small
  \begin{tabular}{ l r@{ \ } r@{ \ } r@{ \ } }
    \toprule
    ~&\multicolumn{1}{c}{\ii{neut.} only} & \multicolumn{1}{c}{30d RNN}  & \multicolumn{1}{c}{50d RNTN}\\
    \midrule
    SICK Train &  56.7 & 95.6 &  \textbf{97.8}  \\
    DG Train &  50.0 & 68.0 & \textbf{74.0}  \\
    Test & 56.7 & 75.3 & \textbf{76.9}  \\
    \bottomrule
  \end{tabular}
  \caption{\mynote{Performance on SICK.}}
  \label{sresultstable}
\end{table} 

The results are shown in Table \ref{sresultstable}. Despite the small amount of high quality training data available and the lack of resources for learning lexical relationships (especially exclusion, as in \ii{cat}--\ii{dog}), it is possible to train our model to perform competitively on textual entailment. Our performance did not reach that of the winning system (84.6\%), but exceeded the performance of eight out of 18 submitted systems including several which used sophisticated hand-engineered features and lexical resources specific to the version of the entailment task at hand.

Table \ref{examplesofsickdata} shows a few typical examples of correct classifications. The model is able to successfully recognize negation and the insertion and deletion of words, as demonstrated in the first two examples. The latter two show that the model can recognize entailments and contradictions between superficially different sentences, such as passive--active pairs and sentences with low lexical overlap.
\section{Discussion and conclusion}\label{sec:discussion}

This paper first evaluates two recursive models on a series of three increasingly
challenging interpretive tasks involving natural language inference---the 
core relational algebra of natural logic with entailment and
exclusion; recursive propositional logic structures; and statements
involving quantification and negation. We then showed that the same models can learn to
perform an entailment task on natural language image captions. \mynote{The results suggest that RNTNs,
and potentially also RNNs}, have the capacity to model these tasks with 
reasonably-sized training sets. These positive results are
promising for the future of learned representation models in the
applied modeling of compositional semantics.

Of course, challenges remain. Even
the RNTN falls short of perfection in the recursion experiment, with
performance falling off steadily as the size of the expressions grows. It
remains to be seen whether these deficiencies can be overcome with
stronger models or optimization procedures. The space of possible stronger
models is sizable, including extensions of the recursive models used here
\cite{sochergrounded,kalchbrenner2014convolutional,irsoydeep}, recurrent
models \cite{sutskever2014sequence}, and even learning-enabled variants 
of more structured models like those of \newcite{grefenstette2013towards} or \newcite{rocktaschellow}.
In addition,
there remain subtle questions about how to fairly assess whether these
models have truly generalized in the way we want them to. There is a
constant tension between showing the models training data that gives
them a chance to learn the target logical functions and revealing the
answer to them in a way that leads to overfitting. The underlying
logical theories provide only limited guidance on this point.

Our artificial data experiments have only scratched the surface 
of the logical complexity of natural language; in future experiments, we hope to test sentences
with embedded quantifiers, multiple interacting quantifiers, relative
clauses, and other kinds of recursive structure. Our SICK experiments 
similarly only begin to reveal the potential of these models to learn to 
perform complex semantic inferences from corpora. Nonetheless, the
rapid progress the field has made with these models in recent years
provides ample reason to be optimistic that they can be trained to
meet all the challenges of natural language semantics.

% These experiments represent one of the first attempts to reproduce any large fragment of the behavior of a complex logic within a neural network model, and the first attempt that we are aware of to address either the encoding of lexical relations or the learning of recursive operators. This presents considerable challenges in evaluating the particular models that we choose, since we cannot rely on prior results to establish that any particular amount or type of training data is sufficient to teach any model the structure of the logic. The positive results that we have found, however, are extremely promising for the future of learned representation models in the applied modeling of meaning. We have seen that recursive neural tensor networks are able to encode lexical relations accurately and encode recursive operators. We have also seen that both RNNs and RNTNs are able to handle the meanings of quantifiers in an inference setting in at least some cases. 

% There is ample room to build on these results. In the interest of fully mirroring the capacity of existing natural logics in learned models, it would be valuable to extend these experiments to cover other ways in which meanings are encoded in natural language, including challenges such as reasoning over sentences with transitive verbs or relative clauses. In addition, it would be highly informative to compare these results on standard recursive neural networks with other proposed learned models for sentence meaning, such as dependency tree RNNs \cite{sochergrounded}, Belief Propagation RNNs (-), or convolutional RNNs \cite{kalchbrenner2014convolutional}.



%\section{Introduction}
%
%
%The following instructions are directed to authors of papers accepted
%for publication in the NAACL HLT 2015 proceedings.  All authors are required
%to adhere to these specifications. Authors are required to provide 
%a Portable Document Format (PDF) version of
%their papers.  The proceedings will be printed on US-Letter paper.
%Authors from countries in which access to word-processing systems is
%limited should contact the publication chairs as soon as possible.
%
%\indent\paragraph{Note} Grayscale readability of all figures and
%graphics will be enforced for all accepted papers
%(\S\ref{ssec:accessibility}).  Apart from this, the style files and
%camera-ready requirements are unchanged from last year.
%
%\section{General Instructions}
%
%Manuscripts must be in two-column format.  Exceptions to the
%two-column format include the title, as well as the 
%authors' names and complete
%addresses (only in the final version, not in the version submitted for review), 
%which must be centered at the top of the first page (see
%the guidelines in Subsection~\ref{ssec:first}), and any full-width
%figures or tables.  Type single-spaced.  Do not number the pages.
%Start all pages directly under the top margin.  See the guidelines
%later regarding formatting the first page.
%
%%% If the paper is produced by a printer, make sure that the quality
%%% of the output is dark enough to photocopy well.  It may be necessary
%%% to have your laser printer adjusted for this purpose.  Papers that are too
%%% faint to reproduce well may not be included.
%
%%% {\bf Do not print page numbers on the manuscript.}  Write them lightly
%%% on the back of each page in the upper left corner along with the
%%% (first) author's name.
%
%The maximum length of a manuscript is eight (8) pages for the main
%conference, printed single-sided, plus two (2) pages for references
%(see Section~\ref{sec:length} for additional information on the
%maximum number of pages).  Do not number the pages.
%
%The review process is double-blind, so do not include any author information (names, addresses) when submitting a paper for review.  However, you should allocate space for the names and addresses so that they will fit in the final (accepted) version.  This is best done by either providing fake or blank names and addresses (as shown in this paper).
%
%\subsection{Electronically-available resources}
%
%NAACL HLT provides this description in \LaTeX2e{} ({\tt naaclhlt2015.tex}) and PDF
%format ({\tt naaclhlt2015.pdf}), along with the \LaTeX2e{} style file used to
%format it ({\tt naaclhlt2015.sty}) and an ACL bibliography style ({\tt naaclhlt2015.bst}).
%These files are all available at
%{\tt http://naacl2015.naacl.org}.  A Microsoft Word
%template file ({\tt naaclhlt2015.dot}) is also available at the same URL. We
%strongly recommend the use of these style files, which have been
%appropriately tailored for the NAACL HLT 2015 proceedings.
%
%
%\subsection{Format of Electronic Manuscript}
%\label{sect:pdf}
%
%For the production of the electronic manuscript you must use Adobe's
%Portable Document Format (PDF). This format can be generated from
%postscript files: on Unix systems, you can use {\tt ps2pdf} for this
%purpose; under Microsoft Windows, you can use Adobe's Distiller, or
%if you have cygwin installed, you can use {\tt dvipdf} or
%{\tt ps2pdf}.  Note 
%that some word processing programs generate PDF which may not include
%all the necessary fonts (esp. tree diagrams, symbols). When you print
%or create the PDF file, there is usually an option in your printer
%setup to include none, all or just non-standard fonts.  Please make
%sure that you select the option of including ALL the fonts.  {\em
%  Before sending it, test your {\/\em PDF} by printing it from a
%  computer different from the one where it was created}. Moreover,
%some word processor may generate very large postscript/PDF files,
%where each page is rendered as an image. Such images may reproduce
%poorly.  In this case, try alternative ways to obtain the postscript
%and/or PDF.  One way on some systems is to install a driver for a
%postscript printer, send your document to the printer specifying
%``Output to a file'', then convert the file to PDF.
%
%For reasons of uniformity, Adobe's {\bf Times Roman} font should be
%used. In \LaTeX2e{} this is accomplished by putting
%
%\begin{quote}
%\begin{verbatim}
%\usepackage{times}
%\usepackage{latexsym}
%\end{verbatim}
%\end{quote}
%in the preamble.
%
%Additionally, it is of utmost importance to specify the {\bf
%  US-Letter format} (8.5in $\times$ 11in) when formatting the paper.
%When working with {\tt dvips}, for instance, one should specify {\tt
%  -t letter}.
%
%Print-outs of the PDF file on US-Letter paper should be identical to the
%hardcopy version.  If you cannot meet the above requirements about the
%production of your electronic submission, please contact the
%publication chairs above  as soon as possible.
%
%
%\subsection{Layout}
%\label{ssec:layout}
%
%Format manuscripts two columns to a page, in the manner these
%instructions are formatted. The exact dimensions for a page on US-letter
%paper are:
%
%\begin{itemize}
%\item Left and right margins: 1 inch
%\item Top margin: 1 inch
%\item Bottom margin: 1 inch
%\item Column width: 3.15 inches
%\item Column height: 9 inches
%\item Gap between columns: 0.2 inches
%\end{itemize}
%
%\noindent Papers should not be submitted on any other paper size. Exceptionally,
%authors for whom it is \emph{impossible} to format on US-Letter paper,
%may format for \emph{A4} paper. In this case, they should keep the \emph{top}
%and \emph{left} margins as given above, use the same column width,
%height and gap, and modify the bottom and right margins as necessary.
%Note that the text will no longer be centered.
%
%\subsection{The First Page}
%\label{ssec:first}
%
%Center the title, author's name(s) and affiliation(s) across both
%columns (or, in the case of initial submission, space for the names). 
%Do not use footnotes for affiliations.  Do not include the
%paper ID number assigned during the submission process. 
%Use the two-column format only when you begin the abstract.
%
%{\bf Title}: Place the title centered at the top of the first page, in
%a 15 point bold font.  (For a complete guide to font sizes and styles, see Table~\ref{font-table}.)
%Long title should be typed on two lines without
%a blank line intervening. Approximately, put the title at 1in from the
%top of the page, followed by a blank line, then the author's names(s),
%and the affiliation on the following line.  Do not use only initials
%for given names (middle initials are allowed). Do not format surnames
%in all capitals (e.g., ``Bangalore,'' not ``BANGALORE'').  The affiliation should
%contain the author's complete address, and if possible an electronic
%mail address. Leave about 0.75in between the affiliation and the body
%of the first page.
%
%{\bf Abstract}: Type the abstract at the beginning of the first
%column.  The width of the abstract text should be smaller than the
%width of the columns for the text in the body of the paper by about
%0.25in on each side.  Center the word {\bf Abstract} in a 12 point
%bold font above the body of the abstract. The abstract should be a
%concise summary of the general thesis and conclusions of the paper.
%It should be no longer than 200 words.  The abstract text should be in 10 point font.
%
%{\bf Text}: Begin typing the main body of the text immediately after
%the abstract, observing the two-column format as shown in 
%the present document.  Do not include page numbers.
%
%{\bf Indent} when starting a new paragraph. For reasons of uniformity,
%use Adobe's {\bf Times Roman} fonts, with 11 points for text and 
%subsection headings, 12 points for section headings and 15 points for
%the title.  If Times Roman is unavailable, use {\bf Computer Modern
%  Roman} (\LaTeX2e{}'s default; see section \ref{sect:pdf} above).
%Note that the latter is about 10\% less dense than Adobe's Times Roman
%font.
%
%\subsection{Sections}
%
%{\bf Headings}: Type and label section and subsection headings in the
%style shown on the present document.  Use numbered sections (Arabic
%numerals) in order to facilitate cross references. Number subsections
%with the section number and the subsection number separated by a dot,
%in Arabic numerals. 
%
%{\bf Citations}: Citations within the text appear
%in parentheses as~\cite{Gusfield:97} or, if the author's name appears in
%the text itself, as Gusfield~\shortcite{Gusfield:97}. In \LaTeX2e, the former is accomplished using
%\verb|\cite| and the latter with \verb|\shortcite| or \verb|\newcite|.
%Append lowercase letters to the year in cases of ambiguities.  
%Treat double authors as in~\cite{Aho:72}, but write as 
%in~\cite{Chandra:81} when more than two authors are involved. 
%Collapse multiple citations as in~\cite{Gusfield:97,Aho:72}.
%
%\textbf{References}: Gather the full set of references together under
%the heading {\bf References}; place the section before any Appendices,
%unless they contain references. Arrange the references alphabetically
%by first author, rather than by order of occurrence in the text.
%Provide as complete a citation as possible, using a consistent format,
%such as the one for {\em Computational Linguistics\/} or the one in the 
%{\em Publication Manual of the American 
%Psychological Association\/}~\cite{APA:83}.  Use of full names for
%authors rather than initials is preferred.  A list of abbreviations
%for common computer science journals can be found in the ACM 
%{\em Computing Reviews\/}~\cite{ACM:83}.
%
%The \LaTeX{} and Bib\TeX{} style files provided roughly fit the
%American Psychological Association format, allowing regular citations, 
%short citations and multiple citations as described above.
%
%{\bf Appendices}: Appendices, if any, directly follow the text and the
%references (but see above).  Letter them in sequence and provide an
%informative title: {\bf Appendix A. Title of Appendix}.
%
%\textbf{Acknowledgment} sections should go as a last (unnumbered) section immediately
%before the references.  
%
%\subsection{Footnotes}
%
%{\bf Footnotes}: Put footnotes at the bottom of the page. They may
%be numbered or referred to by asterisks or other
%symbols.\footnote{This is how a footnote should appear.} Footnotes
%should be separated from the text by a line.\footnote{Note the
%line separating the footnotes from the text.}  Footnotes should be in 9 point font.
%
%\subsection{Graphics}
%
%{\bf Illustrations}: Place figures, tables, and photographs in the
%paper near where they are first discussed, rather than at the end, if
%possible.  Wide illustrations may run across both columns and should be placed at
%the top of a page. Color illustrations are discouraged, unless you have verified that 
%they will be understandable when printed in black ink. 
%
%\begin{table}
%\begin{center}
%\begin{tabular}{|l|rl|}
%\hline \bf Type of Text & \bf Font Size & \bf Style \\ \hline
%paper title & 15 pt & bold \\
%author names & 12 pt & bold \\
%author affiliation & 12 pt & \\
%the word ``Abstract'' & 12 pt & bold \\
%section titles & 12 pt & bold \\
%document text & 11 pt  &\\
%abstract text & 10 pt & \\
%captions & 10 pt & \\
%bibliography & 10 pt & \\
%footnotes & 9 pt & \\
%\hline
%\end{tabular}
%\end{center}
%\caption{\label{font-table} Font guide. }
%\end{table}
%
%{\bf Captions}: Provide a caption for every illustration; number each one
%sequentially in the form:  ``Figure 1. Caption of the Figure.'' ``Table 1.
%Caption of the Table.''  Type the captions of the figures and 
%tables below the body, using 10 point text.  
%
%\subsection{Accessibility}
%\label{ssec:accessibility}
%
%In an effort to accommodate the color-blind (as well as those printing
%to paper), grayscale readability for all accepted papers will be
%enforced.  Color is not forbidden, but authors should ensure that
%tables and figures do not rely solely on color to convey critical
%distinctions.
%
%\section{Length of Submission}
%\label{sec:length}
%
%The NAACL HLT 2015 main conference accepts submissions of long papers
%and short papers.  The maximum length of a long paper manuscript is
%eight (8) pages of content and two (2) additional pages of references
%\emph{only} (appendices count against the eight pages, not the
%additional two pages).  The maximum length of a short paper manuscript
%is four (4) pages and two (2) additional pages of references.
%Accepted papers will be granted an additional content page. For both
%long and short papers, all illustrations, references, and appendices
%must be accommodated within these page limits, observing the
%formatting instructions given in the present document.  Papers that do
%not conform to the specified length and formatting requirements are
%subject to be rejected without review.
%
%\section{Double-blind review process}
%\label{sec:blind}
%
%As the reviewing will be blind, the paper must not include the
%authors' names and affiliations.  Furthermore, self-references that
%reveal the author's identity, e.g., ``We previously showed (Smith,
%1991) ...'' must be avoided. Instead, use citations such as ``Smith
%previously showed (Smith, 1991) ...'' Papers that do not conform to
%these requirements will be rejected without review. In addition,
%please do not post your submissions on the web until after the
%review process is complete (in special cases this is permitted: see 
%the multiple submission policy below).
%
%We will reject without review any papers that do not follow the
%official style guidelines, anonymity conditions and page limits.
%
%\section{Multiple Submission Policy}
%
%Papers that have been or will be submitted to other meetings or
%publications must indicate this at submission time. Authors of
%papers accepted for presentation at NAACL HLT 2015 must notify the
%program chairs by the camera-ready deadline as to whether the paper
%will be presented. All accepted papers must be presented at the
%conference to appear in the proceedings. We will not accept for
%publication or presentation papers that overlap significantly in
%content or results with papers that will be (or have been) published
%elsewhere.
%
%Preprint servers such as arXiv.org and ACL-related workshops that
%do not have published proceedings in the ACL Anthology are not
%considered archival for purposes of submission. Authors must state
%in the online submission form the name of the workshop or preprint
%server and title of the non-archival version.  The submitted version
%should be suitably anonymized and not contain references to the
%prior non-archival version. Reviewers will be told: ``The author(s)
%have notified us that there exists a non-archival previous version
%of this paper with significantly overlapping text. We have approved
%submission under these circumstances, but to preserve the spirit
%of blind review, the current submission does not reference the
%non-archival version.'' Reviewers are free to do what they like with
%this information.
%
%Authors submitting more than one paper to NAACL HLT must ensure
%that submissions do not overlap significantly ($>25\%$) with each other
%in content or results. Authors should not submit short and long
%versions of papers with substantial overlap in their original
%contributions.

%%%

% \section*{Acknowledgments}

% Do not number the acknowledgment section.

\bibliographystyle{naaclhlt2015}
\bibliography{MLSemantics} 

\end{document}
