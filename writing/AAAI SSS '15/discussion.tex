\subsection*{General discussion}\label{sec:discussion}
% TODO: Rewrite

This paper evaluated two neural models on the task of learning diverse
natural logic relations between distributed word representations. The
results suggest that NTNs, but not plain NNs, have the capacity to
meet this challenge with reasonably-sized training sets. In
\cite{Bowman:Potts:Manning:2014}, we extend these results to include
complex expressions involving logical connectives, with similar
conclusions about (recursive versions of) these models. These findings
are promising for the future of learned representation models in the
applied modeling of logical semantics.

% Of course, challenges remain. In terms of our experimental data, even
% the RNTN falls short of perfection in our more complex tasks, with
% performance falling off steadily as the depth of recursion grows. It
% remains to be seen whether these deficiencies can be overcome with
% improvements to the model, the optimization procedures, or the
% linguistic representations
% \cite{sochergrounded,kalchbrenner2014convolutional}. In addition,
% there remain subtle questions about how to fairly assess whether these
% models have truly generalized in the way we want them to. There is a
% constant tension between showing the models training data that gives
% them a chance to learn the target logical functions and revealing the
% answer to them in a way that leads to overfitting. The underlying
% logical theories provide only limited guidance on this point.
%
%, and the fact that there is a finite universe of possible
%  expressions makes his an unavoidable issue. 
%
%  CP: I don't understand the above. It is false if taken literally;
%  our PL generates an infinte number of formulae.
%
% Finally, we have only scratched the surface of the logical complexity
% of natural language; in future experiments, we hope to test sentences
% with embedded quantifiers, multiple interacting quantifiers, relative
% clauses, and other kinds of recursive structure. Nonetheless, the
% rapid progress the field has made with these models in recent years
% provides ample reason to be optimistic that they can be trained to
% meet the challenges of natural language semantics.

% There may also be value in exploring the degree to which this type
% of high-precision representation learning is possible when the set
% of possible relations is large, as in work on embedding representations
% for knowledge base population \cite{riedel2013relation}.

% These experiments represent one of the first attempts to reproduce any large fragment of the behavior of a complex logic within a neural network model, and the first attempt that we are aware of to address either the encoding of lexical relations or the learning of recursive operators. This presents considerable challenges in evaluating the particular models that we choose, since we cannot rely on prior results to establish that any particular amount or type of training data is sufficient to teach any model the structure of the logic. The positive results that we have found, however, are extremely promising for the future of learned representation models in the applied modeling of meaning. We have seen that recursive neural tensor networks are able to encode lexical relations accurately and encode recursive operators. We have also seen that both RNNs and RNTNs are able to handle the meanings of quantifiers in an inference setting in at least some cases. 

% There is ample room to build on these results. In the interest of fully mirroring the capacity of existing natural logics in learned models, it would be valuable to extend these experiments to cover other ways in which meanings are encoded in natural language, including challenges such as reasoning over sentences with transitive verbs or relative clauses. In addition, it would be highly informative to compare these results on standard recursive neural networks with other proposed learned models for sentence meaning, such as dependency tree RNNs \cite{sochergrounded}, Belief Propagation RNNs (TODO: cite), or convolutional RNNs \cite{kalchbrenner2014convolutional}.

