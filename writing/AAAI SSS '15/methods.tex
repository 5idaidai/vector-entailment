
\section{Neural network models for relation classification} \label{methods}

We build embedding-based models using the method of
\cite{Bowman:Potts:Manning:2014}, which is centered on the task of
labeling a pair of words or sentences with one of a small set of
logical relations. The architecture that we use, which is limited to
only pairs of single terms (such as words), is depicted in
Figure~\ref{sample-figure}. The model represents the two input terms
as embeddings, which are fed into a comparison function based on one
of two types of neural network layer functions to produce a
representation for the relationship between the two terms. This
representation is then fed into a softmax classifier, which outputs a
distribution over possible labels. The entire network, including the
embeddings, is trained using backpropagation.

The simpler version of the comparison concatenates the two
input vectors before feeding them into a standard neural network (NN)
layer.  The more powerful neural tensor network (NTN) version uses an
additional third-order tensor parameter to allow for multiplicative
interactions between the two inputs \cite{chen2013learning}. For more
details on the implementation and training of the layer functions, see
\cite{Bowman:Potts:Manning:2014}.

\begin{figure}[tp]
  \centering
  \footnotesize

\newcommand{\labeledtreenode}[4][3.5]{\put(#2){\makebox(0,0){{\fcolorbox{black}{#4}{\makebox(#1,0.3){#3}}}}}}

\newcommand{\textlabel}[4][3.5]{\put(#2){\makebox(0,0){{\fcolorbox{white}{white}{\makebox(#1,0.3){#3}}}}}}

\definecolor{lexcolor}{HTML}{F5F7C4}
\definecolor{compositioncolor}{HTML}{BBEBFF}
\definecolor{comparisoncolor}{HTML}{FFC895}
\definecolor{softmaxcolor}{HTML}{A5FF8A}


\setlength{\unitlength}{0.61cm}
\begin{picture}(21,7.5)
  
  \labeledtreenode[2.4]{11.5,7}{$P(\sqsubset) = 0.8$}{softmaxcolor}  
  \put(11.5,5.7){\vector(0,1){1}}  
  \labeledtreenode[7.85]{11.5,5.4}{all reptiles walk \emph{vs.}~some turtles move}{comparisoncolor}


  \textlabel{8,7}{Softmax classifier}{black}
  \textlabel{4.5,5.4}{Comparison N(T)N layer}{black}
      
  \textlabel{11.75,3.6}{Composition RN(T)N layers}{black}

  \textlabel{5,0.1}{Learned, randomly initialized word vectors}{black}
  
  %%%%%%%%%%%%%%%%%%%%%%%%%%%%%%%%%%%%%%%%%%%%%%%%%%
    
  \put(1.75,1.35){\vector(2,1){1.7}}
  \labeledtreenode{1.75,1}{all}{lexcolor}

  \put(6,1.35){\vector(-2,1){1.7}}
  \labeledtreenode{6,1}{reptiles}{lexcolor}

  \put(4,2.75){\vector(2,1){1.7}}
  \labeledtreenode{4,2.5}{all reptiles}{compositioncolor}

  \put(8.25,2.75){\vector(-2,1){1.7}}
  \labeledtreenode{8.25,2.5}{walk}{lexcolor}

  \put(6.25,4.25){\vector(4,1){3.25}}
  \labeledtreenode{6.25,3.9}{all reptiles walk}{compositioncolor}
  
  %%%%%%%%%%%%%%%%%%%%%%%%%%%%%%%%%%%%%%%%%%%%%%%%%%%

  \put(12.75,1.35){\vector(2,1){1.7}}
  \labeledtreenode{12.75,1}{some}{lexcolor}

  \put(17,1.35){\vector(-2,1){1.7}}
  \labeledtreenode{17,1}{turtles}{lexcolor}

  \put(15,2.75){\vector(2,1){1.7}}
  \labeledtreenode{15,2.5}{some turtles}{compositioncolor}

  \put(19.25,2.75){\vector(-2,1){1.7}}
  \labeledtreenode{19.25,2.5}{move}{lexcolor}
          
  \put(17.25,4.25){\vector(-4,1){3.25}}
  \labeledtreenode{17.25,3.9}{some turtles move}{compositioncolor}
  
\end{picture}



  \caption{The model structure used to compare \ii{turtle} and \ii{animal}.} 
  \label{sample-figure}
\end{figure}

This model used here differs from the one described in that work in
two ways. First, because the inputs are single terms, we do not use
a composition function here. Second, for our experiment on WordNet data, 
we introduce an additional neural network layer between the embedding 
input and the comparison function, which facilitates initializing the 
embeddings themselves from pretrained vectors and was found to help 
performance in that setting.

%\ii{Source code and generated data will be released after the conclusion of the review period.} % TODO: Or upon request? Attach anonymized code?

