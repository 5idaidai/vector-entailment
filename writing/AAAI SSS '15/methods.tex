
\subsection*{Neural network models for relation classification} \label{methods}

% TODO: Should we cite our NIPS manuscript?

We follow the approach to learning semantically meaningful embeddings 
proposed in \citet{bowman2013can}, which is centered on the problem of
labeling a pair of words or sentences with one of a small set of logical
relations. The architecture of the model that we use, which is limited
to only pairs of single symbols (such as words), is depicted in
Figure~\ref{sample-figure}. The model represents the two input symbols
as embeddings, which are fed into a comparison function based on one
of two types of neural network layer functions to produce a representation
for the relationship between the two symbols. This representation is then
fed into a simple softmax classifier which outputs a distribution over
possible labels. The entire network, including the embeddings, are trained
through backpropagation with AdaGrad \cite{duchi2011adaptive}.

\begin{figure}[tp]
  \centering
  \footnotesize

\newcommand{\labeledtreenode}[4][3.5]{\put(#2){\makebox(0,0){{\fcolorbox{black}{#4}{\makebox(#1,0.3){#3}}}}}}

\newcommand{\textlabel}[4][3.5]{\put(#2){\makebox(0,0){{\fcolorbox{white}{white}{\makebox(#1,0.3){#3}}}}}}

\definecolor{lexcolor}{HTML}{F5F7C4}
\definecolor{compositioncolor}{HTML}{BBEBFF}
\definecolor{comparisoncolor}{HTML}{FFC895}
\definecolor{softmaxcolor}{HTML}{A5FF8A}


\setlength{\unitlength}{0.61cm}
\begin{picture}(21,7.5)
  
  \labeledtreenode[2.4]{11.5,7}{$P(\sqsubset) = 0.8$}{softmaxcolor}  
  \put(11.5,5.7){\vector(0,1){1}}  
  \labeledtreenode[7.85]{11.5,5.4}{all reptiles walk \emph{vs.}~some turtles move}{comparisoncolor}


  \textlabel{8,7}{Softmax classifier}{black}
  \textlabel{4.5,5.4}{Comparison N(T)N layer}{black}
      
  \textlabel{11.75,3.6}{Composition RN(T)N layers}{black}

  \textlabel{5,0.1}{Learned, randomly initialized word vectors}{black}
  
  %%%%%%%%%%%%%%%%%%%%%%%%%%%%%%%%%%%%%%%%%%%%%%%%%%
    
  \put(1.75,1.35){\vector(2,1){1.7}}
  \labeledtreenode{1.75,1}{all}{lexcolor}

  \put(6,1.35){\vector(-2,1){1.7}}
  \labeledtreenode{6,1}{reptiles}{lexcolor}

  \put(4,2.75){\vector(2,1){1.7}}
  \labeledtreenode{4,2.5}{all reptiles}{compositioncolor}

  \put(8.25,2.75){\vector(-2,1){1.7}}
  \labeledtreenode{8.25,2.5}{walk}{lexcolor}

  \put(6.25,4.25){\vector(4,1){3.25}}
  \labeledtreenode{6.25,3.9}{all reptiles walk}{compositioncolor}
  
  %%%%%%%%%%%%%%%%%%%%%%%%%%%%%%%%%%%%%%%%%%%%%%%%%%%

  \put(12.75,1.35){\vector(2,1){1.7}}
  \labeledtreenode{12.75,1}{some}{lexcolor}

  \put(17,1.35){\vector(-2,1){1.7}}
  \labeledtreenode{17,1}{turtles}{lexcolor}

  \put(15,2.75){\vector(2,1){1.7}}
  \labeledtreenode{15,2.5}{some turtles}{compositioncolor}

  \put(19.25,2.75){\vector(-2,1){1.7}}
  \labeledtreenode{19.25,2.5}{move}{lexcolor}
          
  \put(17.25,4.25){\vector(-4,1){3.25}}
  \labeledtreenode{17.25,3.9}{some turtles move}{compositioncolor}
  
\end{picture}



  \caption{The model structure used to compare \ii{turtle} and \ii{animal}. 
    The same structure is used for both the RNN and RNTN layer functions.} 
  \label{sample-figure}
\end{figure}

For a comparison funtion, we evaluate versions of the model with both a plain neural
network (NN) layer function and a neural tensor network (NTN) layer function
\eqref{rntn} proposed in \citet{chen2013learning}. A leaky ReLU
nonlinearity \cite{maasrectifier} is applied to the output of either
layer function.
%
\begin{gather} \label{rnn}
\vec{y}_{\textit{NN}} = f(\mathbf{M} [\vec{x}^{(l)}; \vec{x}^{(r)}] + \vec{b}) \\ % TODO: Add column vectors?
\label{rntn}
\vec{y}_{\textit{NTN}} = f(\vec{x}^{(l)T} \mathbf{T}^{[1 \ldots n]} \vec{x}^{(r)} + \mathbf{M} [\vec{x}^{(l)}; \vec{x}^{(r)}] + \vec{b})
\end{gather} 
%
Here, $\vec{x}^{(l)}$ and $\vec{x}^{(r)}$ are the column vector
representations for the left and right children of the node, and
$\vec{y}$ is the node's output.  The RNN concatenates them, multiplies
them by an $n \times 2n$ matrix of learned weights, and applies the
element-wise non-linearity to the resulting vector. The RNTN has the
same basic structure, but with the addition of a learned third-order
tensor $\mathbf{T}$, dimension $n \times n \times n$, modeling
multiplicative interactions between the child vectors. Both models
include a bias vector~$\vec{b}$.


This model differs from that of \citet{bowman2013can} in two ways. Because 
the inputs are single symbols, there is no need for the composition functions
which are used in that (and prior) work. Also, for our second experiment on 
WordNet data, we introduce a new neural network layer between the embedding input
and the comparison function, which is meant to facilitate initializing the embeddings
from an outside source, and was found to help peformance in that setting.

%\ii{Source code and generated data will be released after the conclusion of the review period.} % TODO: Or upon request? Attach anonymized code?

